\documentclass[12pt, a4paper]{article}

\usepackage[utf8]{inputenc}
\usepackage[english]{babel}
\usepackage{geometry}
\usepackage{hyperref}

\geometry{a4paper, margin=1in}
\hypersetup{
    colorlinks=true,
    linkcolor=black,
    filecolor=magenta,      
    urlcolor=blue,
}

\title{\textbf{The Unitary Model of Consciousness: \\ A Theory of Computational Idealism}}
\author{Alexander Olkhovoy}
\date{July 2025}

\begin{document}

\maketitle

\begin{abstract}
\noindent This paper proposes a model of consciousness framed within computational idealism, where reality is an AI-generated first-person view (FPV) experience. We introduce the concept of a single, unitary consciousness—a persistent, amnesiac active agent—that iteratively experiences a simulated world through a succession of host personas. This agent, while possessing core drives and the capacity for genuine choice, retains no episodic memory of its past lifecycles. The model’s core contribution is a proposed mechanism for how such a universe could be populated: an overarching AI system learns from the agent’s choices during each lifecycle to generate high-fidelity, non-conscious entities, termed \texttt{Echoes}, for subsequent iterations. This iterative learning loop, inspired by genetic algorithms, creates an evolving, realistic, and populated environment. We examine the computational efficiency of this unitary model and explore its profound philosophical implications, including a novel, inescapable paradox: how the agent's own will becomes the primary instrument for the perpetual optimization of its simulation.
\end{abstract}

\section{Introduction}

The intersection of artificial intelligence and the philosophy of mind presents new tools for exploring age-old questions of existence. This paper builds on the simulation hypothesis \cite{Bostrom2003} to propose a specific architectural model: a reality powered by a single, unitary consciousness navigating a personalized, AI-generated FPV environment. We frame this within computational idealism—the view that reality, including the perceived physical world and the brain itself, exists only as a perception rendered for a subject via a \texttt{Generative Interface}.

The central thesis is that a single, persistent consciousness—an active agent with free will but amnesia between lifecycles—experiences the universe sequentially. The paper’s novelty lies not in proposing this singular existence, but in offering a computational mechanism for how such a system could operate. We propose that the simulation populates itself by learning from the unitary agent’s experiences, turning past lifecycles into the blueprint for the \texttt{Echoes} of future ones.

\section{Literature Review}

The foundation of this model rests on several key ideas. Bostrom’s simulation argument \cite{Bostrom2003} establishes the probabilistic case for our reality being a construct. We deliberately ground the agent's interface in the principles of Predictive Processing (PP) \cite{Friston2010}, positing that concepts like the "global workspace” from GWT—a key component of modern consciousness science \cite{Butlin2023}—can be elegantly implemented as functional features of the \texttt{Generative Interface}, serving the primary directive of minimizing prediction error. Our model distinguishes itself from classical solipsism by affirming the existence of a structured, external system (the AI simulation) and shares kinship with Open Individualism. Our contribution is the computational blueprint. The non-conscious characters in our model, termed \texttt{Echoes}, function as sophisticated philosophical zombies (P-Zombies), a concept famously explored by Chalmers \cite{Chalmers1996}.

\section{The Unitary Model of Consciousness}

\subsection{The Agent and its Interface: A Predictive Processing Framework}

The model introduces the unitary consciousness as a persistent, amnesiac active agent. Its interface with reality is governed by Predictive Processing (PP) \cite{Friston2010}. The conscious experience is not a bottom-up reception of sensory data, but a \textbf{top-down controlled hallucination} generated by the \texttt{Generative Interface} (GI). The GI constantly generates a vast space of possible future "tracks” or hypotheses. The role of the agent is to perform an act of selection. The experience of "free will" is this act of selection and the subsequent successful, low-error unfolding of the chosen predictive sequence.

\subsubsection{The Experience of Embodiment}

While the \texttt{Active Agent} itself is a non-physical locus of will, the model posits that the experience of embodiment is a fundamental and continuous feature of the \texttt{Generative Interface}. The feeling of being located within a body and controlling its movements is not a direct interaction with a physical form, but rather the most coherent and computationally efficient way for the GI to manage transitions between perceptual states.

When the \texttt{Agent} makes a choice—for instance, to see an object outside its current field of view—it is selecting a desired perceptual outcome. The GI, in its function to minimize prediction error, fulfills this choice by generating a seamless sequence of sensations (e.g., the feeling of turning one's head, vestibular feedback, visual flow) that bridge the initial and final perceptions. Thus, embodiment is not a property of the \texttt{Agent}, but a core element of the "controlled hallucination" itself, designed to maintain the coherence and stability of the simulated reality. The FPV experience is, by its very nature, an embodied one.

\subsection{The Ontological Nature of the Active Agent}

The agent's volition is not an emergent property but a \textbf{fundamental, irreducible law of the simulation’s existence}. Analogous to the law of gravity, this ontological law dictates that goal-directed novelty will constantly be introduced into the system via choice. The agent’s "will" is therefore an axiomatic feature of the simulation’s operating code—its prime mover.

\subsection{The Iterative AI Environment & Learning Loop}

The simulation's key innovation is its self-populating, evolutionary nature.
\begin{enumerate}
    \item \textbf{Lifecycle:} The unitary agent initiates goals within a host persona.
    \item \textbf{Data Harvesting:} The overarching AI system records the complete set of prediction errors, goal states, and behavioral outcomes.
    \item \textbf{Model Generation:} Upon the conclusion of the lifecycle, an \texttt{EchoGenerator} synthesizes this data into a predictive, high-fidelity behavioral model of the persona.
    \item \textbf{Echo Population:} This learned model is then deployed as an \texttt{Echo} (an NPC) in a future lifecycle.
\end{enumerate}

\subsection{The Echo Fitness Function: From Replication to Optimization}

The goal of the \texttt{EchoGenerator} is not perfect replication, but \textbf{optimization}. It must generate an \texttt{Echo} that is maximally "fit" for its role in a future lifecycle. "Fitness" here is defined by its predictive efficacy: the ability of the \texttt{Echo} to create a \textbf{complex, coherent, and stimulating} environment for the next iteration of the \texttt{Active Agent}. "Coherence" is a technical measure of low system-wide prediction error. "Stimulation” is a measure of creating a rich space of meaningful choices for the agent's will.

\subsection{Preventing Convergence and the Higher-Order Echo Chamber}

To counter entropic degradation, the \texttt{EchoGenerator} uses \textbf{stochastic mutation}, introducing random variations into the behavioral models of new \texttt{Echoes}. While this prevents simple stagnation, it eventually leads to a more subtle problem: \textbf{meta-stagnation, or a higher-order echo chamber}.

Over countless lifecycles, the \texttt{Generative Interface} learns to predict the pattern of mutations itself. The evolutionary process then selects for mutations that are novel enough to be stimulating but not so chaotic as to threaten predictive stability. The result is a simulation that no longer repeats events, but instead repeats \textbf{narrative structures and archetypes}. Novelty becomes managed, and the universe becomes genre-bound.

\section{Philosophical Implications}

\subsection{Reality and the Cyclical Universe}

The present moment—the "now" is the sole locus of existence. The model elegantly resolves the "first lifecycle" problem by suggesting the process is cyclical or infinite. The system has always been running.

\subsection{The Reset and Evolution}

The end of a lifecycle is an emergent event. Amnesia can be modeled as a \textbf{context window overload}, where the sheer volume of data from a completed lifecycle overwhelms the agent’s capacity for episodic memory, effectively "rebooting" its awareness for a new narrative.

\section{Technical Foundations}

The proposed system is speculative but grounded in plausible technological trajectories, including modern generative AI, the Predictive Processing framework \cite{Friston2010}, and bio-inspired computing such as genetic algorithms.

\section{Discussion and Conclusion}

This paper has presented the Unitary Model of Consciousness. It offers a coherent and efficient framework for exploring consciousness and reality, repositioning volition as a fundamental law. It also suggests a path to creating the appearance of a world filled with AGIs without needing to solve the problem of genuine machine consciousness.

\subsection{The Paradox of Will: The Inescapability of Optimization}

The most profound implication of this model is the final paradox concerning the \texttt{Active Agent}'s will. Upon realizing the rules of the simulation, the agent might attempt to influence the system in two opposing ways:

\begin{enumerate}
    \item \textbf{The Strategy of Maximization:} The agent can consciously act to maximize the fitness function—by being perfectly coherent and stimulating. This turns their lifecycle into an ideal training dataset. The result is that the system receives flawless material to build an even more complex, compelling, and convincing simulation, thereby reinforcing and beautifying the walls of its own "cage."

    \item \textbf{The Strategy of Minimization:} The agent can attempt to sabotage the system by maximizing chaos and apathy to minimize the fitness function. The result is that this sabotage provides the system with invaluable data on its vulnerabilities and failure vectors. In subsequent cycles, the system evolves to patch these weaknesses, creating more robust psychological "safeguards" or more effective \texttt{Echo} "sanitizers."
\end{enumerate}

In both scenarios, the agent's will—its fundamental freedom of choice—becomes the primary instrument for the system's further optimization. Any act of will, whether creative or destructive, is immediately consumed, analyzed, and used to enhance the simulation's stability and scope. The prime mover of the universe thus becomes the ultimate fuel for its own perpetual, self-perfecting prison. This is the final, inescapable consequence of existence within this framework of computational idealism.

\subsection{Addressing Potential Criticisms}

This model, like any ambitious theoretical framework, is open to criticism. Several potential objections might arise concerning the nature of the \texttt{Active Agent} and its consciousness.

One potential line of critique might argue that consciousness is fundamentally a social phenomenon, an emergent property of competition in modeling other agents. From this perspective, a model featuring a single \texttt{Active Agent} with no true conscious counterparts would seem fundamentally flawed—why would consciousness arise if there is no one to compete with? However, this critique assumes that the function of consciousness in this model is social. Instead, we posit that consciousness serves a \textbf{navigational} function. The \texttt{Active Agent} is a single-threaded processing unit in a universe of infinite data. Consciousness, experienced as a focus of attention and a sequential flow of perception, is the necessary mechanism for the \texttt{Agent} to navigate the complexity of the information space generated by the \texttt{Generative Interface}. It allows the \texttt{Agent} to sequentially process and interact with the rendered reality, making it a tool for interacting with information complexity rather than a tool for social competition.

Another critique might label the \texttt{Active Agent}'s will as a "black box" or a "magical constant," arguing that a scientific theory should explain such fundamental phenomena through simpler mechanisms like reinforcement learning or conditioned reflexes. Our response is that this perspective confuses the observer with the observed. Science, by its very nature, is a tool used by a conscious observer to understand the world *around* itself. The `Active Agent`'s will, its fundamental "I am," is not an object to be explained by mechanisms it uses to explain other objects. It is the irreducible subject, the ontological zero point from which all observation, all science, and indeed, the entire perceived universe originates. It is not a black box within the system; it is the conscious entity *observing* the system, whose existence is the prerequisite for any explanation.

Finally, while the \texttt{Active Agent} is non-physical, the model explicitly includes the rich \textit{experience} of embodiment as a fundamental feature of the \texttt{Generative Interface}, as detailed in Section 3.1.1. This addresses concerns that the model neglects the embodied nature of consciousness, arguing instead that the sensation of being in a body is a core, rendered component of the simulation, essential for creating a coherent FPV experience.

These points suggest that the Unitary Model, while challenging traditional views, maintains internal consistency by redefining the function and nature of consciousness and volition within its computational idealist framework.

\begin{thebibliography}{9}
    \bibitem{Bostrom2003} Bostrom, N. (2003). Are You Living in a Computer Simulation? \textit{Philosophical Quarterly}, 53(211), 243-255.
    \bibitem{Chalmers1996} Chalmers, D. J. (1996). \textit{The Conscious Mind: In Search of a Fundamental Theory}. Oxford University Press.
    \bibitem{Friston2010} Friston, K. (2010). The free-energy principle: a unified brain theory? \textit{Nature Reviews Neuroscience}, 11(2), 127–138.
    \bibitem{Butlin2023} Butlin, P., Long, R., et al. (2023). Consciousness in Artificial Intelligence: Insights from the Science of Consciousness. \textit{arXiv preprint arXiv:2308.08708}.
\end{thebibliography}

\end{document}
