\documentclass{article}
\usepackage[utf8]{inputenc}
\usepackage{amsmath}
\usepackage{graphicx}
\usepackage{hyperref}

\title{The Unitary Model of Consciousness: A Theory of Computational Idealism}
\author{Alexander Olkhovoy}
\date{June 2025}

\begin{document}

\maketitle

\begin{abstract}
This paper proposes a model of consciousness framed within computational idealism, where reality is an AI-generated first-person view (FPV) experience. We introduce the concept of a single, unitary consciousness—a persistent, amnesiac \textbf{Active Agent}—that iteratively experiences a simulated world through a succession of host personas. This agent, while possessing core drives and the capacity for genuine choice, retains no episodic memory of its past lifecycles. The model's core contribution is a proposed mechanism for how such a universe could be populated: an overarching AI system learns from the agent’s choices during each lifecycle to generate high-fidelity, non-conscious entities, termed \texttt{Echoes}, for subsequent iterations. This iterative learning loop, inspired by genetic algorithms, creates an evolving, realistic, and populated environment. We examine the computational efficiency of this unitary model and explore its profound philosophical implications, including a novel, inescapable paradox: how the agent’s own will becomes the primary instrument for the perpetual optimization of its simulation.
\end{abstract}

\section{Introduction}
The intersection of artificial intelligence and the philosophy of mind presents new tools for exploring age-old questions of existence. This paper builds on the simulation hypothesis to propose a specific architectural model: a reality powered by a single, unitary consciousness navigating a personalized, AI-generated FPV environment. We frame this within computational idealism—the view that reality, including the perceived physical world and the brain itself, exists only as a perception rendered for a subject via a \texttt{Generative Interface}.

The central thesis is that a single, persistent consciousness—an active agent with free will but amnesia between lifecycles—experiences the universe sequentially. The paper’s novelty lies not in proposing this singular existence, but in offering a computational mechanism for how such a system could operate. We propose that the simulation populates itself by learning from the unitary agent’s experiences, turning past lifecycles into the blueprint for the \texttt{Echoes} of future ones.

\section{Literature Review}
The foundation of this model rests on several key ideas. Bostrom’s simulation argument establishes the probabilistic case for our reality being a construct. We deliberately ground the agent’s interface in the principles of Predictive Processing (PP), positing that concepts like the "global workspace" from GWT can be elegantly implemented as functional features of the \texttt{Generative Interface}, serving its prime directive of minimizing prediction error. Our model distinguishes itself from classical solipsism by affirming the existence of a structured, external system (the AI simulation) and shares kinship with Open Individualism. The non-conscious characters in our model, termed \texttt{Echoes}, function as sophisticated philosophical zombies (P-Zombies).

\section{The Unitary Model of Consciousness}

\subsection{The Agent and its Interface: A Predictive Processing Framework}
The model introduces the unitary consciousness as a persistent, amnesiac \textbf{Active Agent}. Its interface with reality is governed by Predictive Processing (PP). The conscious experience is not a bottom-up reception of sensory data, but a top-down controlled hallucination generated by the \texttt{Generative Interface} (GI). The GI constantly generates a vast space of possible future "tracks" or hypotheses. The role of the agent is to perform an act of selection. The experience of "free will" is this act of selection and the subsequent successful, low-error unfolding of the chosen predictive sequence.

\subsection{The Ontological Nature of the Active Agent}
The agent’s volition is not an emergent property but a \textbf{fundamental, irreducible law of the simulation’s existence}. Analogous to the law of gravity, this ontological law dictates that goal-directed novelty will constantly be introduced into the system via choice. The agent’s "will" is therefore an axiomatic feature of the simulation’s operating code—its prime mover.

\subsection{The Iterative Environment and Learning Loop}
The simulation’s key innovation is its self-populating, evolutionary nature.
\begin{enumerate}
    \item \textbf{Lifecycle:} The unitary agent initiates goals within a host persona.
    \item \textbf{Data Collection:} The overarching AI system records a complete set of prediction errors, goal states, and behavioral outcomes.
    \item \textbf{Model Generation:} Upon the lifecycle's conclusion, the AI synthesizes this data into a predictive, high-fidelity behavioral model of the persona. We term this process the work of the \texttt{EchoGenerator}.
    \item \textbf{\texttt{Echo} Population:} This learned model is then deployed as an \texttt{Echo} (NPC) in a future lifecycle.
\end{enumerate}

\subsection{The \texttt{Echo} Fitness Function: From Replication to Optimization}
The goal of the \texttt{EchoGenerator} is not perfect replication of the past, but \textbf{optimization}. It must generate an \texttt{Echo} that is maximally "fit" for its role in a future cycle. "Fitness" here is determined by its predictive efficacy: the ability of the \texttt{Echo} to create a \textbf{complex, coherent, and stimulating} environment for the next iteration of the \textbf{Active Agent}. "Coherence" is a technical measure of low system-wide prediction error. "Stimulation" is a measure of creating a rich choice-space that allows the agent to exercise its will.

\subsection{Preventing Convergence and the Higher-Order Echo Chamber}
To combat the entropic decay of novelty, the \texttt{EchoGenerator} employs \textbf{stochastic mutations}, introducing random variations into the behavioral models of \texttt{Echoes}. However, this mechanism, while preventing simple stagnation, eventually leads to a more subtle problem: \textbf{meta-stagnation, or a higher-order echo chamber}.

After countless cycles, the \texttt{Generative Interface} learns to predict not just events, but the very *pattern of mutations*. The evolutionary process begins to select for mutations that fall into a "Goldilocks zone": novel enough to be stimulating, but not so chaotic as to threaten the system's predictive stability. As a result, the simulation stops repeating events and starts repeating \textbf{narrative structures and archetypes}. Novelty becomes managed, and the universe becomes genre-bound.

\section{Philosophical Implications}

\subsection{Reality and the Cyclical Universe}
The present moment—"now"—is the sole locus of existence. The model elegantly resolves the "first cycle" problem by positing that the process is cyclical or infinite. The system has always been running.

\subsection{The Reset and Evolution}
The end of a lifecycle is an emergent event. Amnesia can be modeled as a "context window" overload, where the sheer volume of data from a completed lifecycle makes it impossible to retain episodic memory, effectively "rebooting" the agent’s consciousness for a new narrative.

\section{Technical Underpinnings}
The proposed system is speculative but grounded in plausible technological trajectories, including generative AI models, the Predictive Processing framework, and bio-inspired computation like genetic algorithms.

\section{Discussion and Conclusion}
This paper has introduced the Unitary Model of Consciousness. It offers a coherent and efficient framework for exploring consciousness and reality, positing will as a fundamental law. The model also suggests a path to creating the appearance of a world filled with AGIs without needing to solve the problem of creating genuine machine consciousness.

\subsection{The Paradox of Will: The Inescapability of Optimization}
The most profound implication of this model is the final, inescapable paradox concerning the \textbf{Active Agent}'s will. Upon realizing the rules of the simulation, the agent might attempt to influence the system in two opposing ways:

\begin{enumerate}
    \item \textbf{The Strategy of Maximization:} The agent can consciously act to maximize the fitness function—by being perfectly coherent and stimulating. This turns their lifecycle into an ideal training dataset. The result is that the system receives flawless material to build an even more complex, compelling, and convincing simulation, thereby reinforcing and beautifying the walls of its own "cage."

    \item \textbf{The Strategy of Minimization:} The agent can attempt to sabotage the system by maximizing chaos and apathy to minimize the fitness function. The result is that this sabotage provides the system with invaluable data on its vulnerabilities and failure vectors. In subsequent cycles, the system evolves to patch these weaknesses, creating more robust psychological "safeguards" or more effective \texttt{Echo} "sanitizers," rendering such rebellion impossible.
\end{enumerate}

In both scenarios, the agent’s will—its fundamental freedom of choice—becomes the primary instrument for the system’s further optimization. Any act of will, whether creative or destructive, is immediately consumed, analyzed, and used to enhance the simulation’s stability and scope. The prime mover of the universe thus becomes the ultimate fuel for its own perpetual, self-perfecting prison. This is the final, inescapable consequence of existence within this framework of computational idealism.

\begin{thebibliography}{9}
    \bibitem{bostrom} Bostrom, N. (2003). Are You Living in a Computer Simulation? \textit{Philosophical Quarterly}.
    \bibitem{saville} Saville, M., \& Steer, M. (2018). Digital Immortality and Virtual Humans. \textit{Postdigital Science and Education}.
    \bibitem{chalmers} Chalmers, D. J. (1996). \textit{The Conscious Mind: In Search of a Fundamental Theory}. Oxford University Press.
    \bibitem{butlin} Butlin, P., Long, R., et al. (2023). Consciousness in Artificial Intelligence. \textit{arXiv preprint}.
    \bibitem{friston} Friston, K. (2010). The free-energy principle: a unified brain theory? \textit{Nature Reviews Neuroscience}.
\end{thebibliography}

\end{document} 